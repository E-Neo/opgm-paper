\section{Related Work}
Although the general property graph matching problem is seldom discussed in previous works,
simple graph isomorphism problem has been widely studied.

\textbf{Graph Query Engines}.

\begin{table*}
  \caption{Graph Query Engines}\label{tab:query_engines}
  \begin{tabular}{llrrrrrr}
    \toprule
        {} & {} & \multicolumn{6}{c}{Support for Queries} \\
        \cline{3-8}
        Name & Platform & Directed & Undirected & Vertex Label & Edge Label & Multi-edges & WHERE clause \\
        \midrule
        STwig\cite{DBLP:journals/pvldb/SunWWSL12}               & distributed & \textcolor{red}{\XSolidBrush} & \textcolor{green}{\Checkmark} & \textcolor{green}{\Checkmark} & \textcolor{red}{\XSolidBrush} & \textcolor{red}{\XSolidBrush} & \textcolor{red}{\XSolidBrush} \\
        PSgL\cite{DBLP:conf/sigmod/ShaoCCMYX14}                 & distributed & \textcolor{red}{\XSolidBrush} & \textcolor{green}{\Checkmark} & \textcolor{red}{\XSolidBrush} & \textcolor{red}{\XSolidBrush} & \textcolor{red}{\XSolidBrush} & \textcolor{red}{\XSolidBrush} \\
        TwinTwig\cite{DBLP:journals/pvldb/LaiQLC15}             & distributed & \textcolor{red}{\XSolidBrush} & \textcolor{green}{\Checkmark} & \textcolor{red}{\XSolidBrush} & \textcolor{red}{\XSolidBrush} & \textcolor{red}{\XSolidBrush} & \textcolor{red}{\XSolidBrush} \\
        SEED\cite{DBLP:journals/pvldb/LaiQLZC16}                & distributed & \textcolor{red}{\XSolidBrush} & \textcolor{green}{\Checkmark} & \textcolor{red}{\XSolidBrush} & \textcolor{red}{\XSolidBrush} & \textcolor{red}{\XSolidBrush} & \textcolor{red}{\XSolidBrush} \\
        QFrag\cite{DBLP:conf/cloud/SerafiniMS17}                & distributed & \textcolor{red}{\XSolidBrush} & \textcolor{green}{\Checkmark} & \textcolor{green}{\Checkmark} & \textcolor{red}{\XSolidBrush} & \textcolor{red}{\XSolidBrush} & \textcolor{red}{\XSolidBrush} \\
        VCBC/CBF\cite{DBLP:journals/pvldb/QiaoZC17}             & distributed & \textcolor{red}{\XSolidBrush} & \textcolor{green}{\Checkmark} & \textcolor{red}{\XSolidBrush} & \textcolor{red}{\XSolidBrush} & \textcolor{red}{\XSolidBrush} & \textcolor{red}{\XSolidBrush} \\
        Fractal\cite{DBLP:conf/sigmod/DiasTGM019}               & distributed & \textcolor{red}{\XSolidBrush} & \textcolor{green}{\Checkmark} & \textcolor{green}{\Checkmark} & \textcolor{red}{\XSolidBrush} & \textcolor{red}{\XSolidBrush} & \textcolor{red}{\XSolidBrush} \\
        \midrule
        Ullmann\cite{DBLP:journals/jacm/Ullmann76}              & in-memory   & \textcolor{red}{\XSolidBrush} & \textcolor{green}{\Checkmark} & \textcolor{red}{\XSolidBrush} & \textcolor{red}{\XSolidBrush} & \textcolor{red}{\XSolidBrush} & \textcolor{red}{\XSolidBrush} \\
        QuickSI\cite{DBLP:journals/pvldb/ShangZLY08}            & in-memory   & \textcolor{red}{\XSolidBrush} & \textcolor{green}{\Checkmark} & \textcolor{green}{\Checkmark} & \textcolor{red}{\XSolidBrush} & \textcolor{red}{\XSolidBrush} & \textcolor{red}{\XSolidBrush} \\
        Turbo\textsubscript{ISO}\cite{DBLP:conf/sigmod/HanLL13} & in-memory   & \textcolor{red}{\XSolidBrush} & \textcolor{green}{\Checkmark} & \textcolor{green}{\Checkmark} & \textcolor{red}{\XSolidBrush} & \textcolor{red}{\XSolidBrush} & \textcolor{red}{\XSolidBrush} \\
        Graphflow\cite{DBLP:journals/pvldb/MhedhbiS19}          & in-memory   & \textcolor{green}{\Checkmark} & \textcolor{red}{\XSolidBrush} & \textcolor{green}{\Checkmark} & \textcolor{green}{\Checkmark} & \textcolor{red}{\XSolidBrush} & \textcolor{red}{\XSolidBrush} \\
        Neo4j                                                   & in-memory   & \textcolor{green}{\Checkmark} & \textcolor{green}{\Checkmark} & \textcolor{green}{\Checkmark} & \textcolor{green}{\Checkmark} & \textcolor{green}{\Checkmark} & \textcolor{green}{\Checkmark} \\
        \midrule
        \textsc{DualSim}\cite{DBLP:conf/sigmod/KimLBHLKJ16}     & out-of-core & \textcolor{red}{\XSolidBrush} & \textcolor{green}{\Checkmark} & \textcolor{red}{\XSolidBrush} & \textcolor{red}{\XSolidBrush} & \textcolor{red}{\XSolidBrush} & \textcolor{red}{\XSolidBrush} \\
        SeqStar                                                 & out-of-core & \textcolor{green}{\Checkmark} & \textcolor{green}{\Checkmark} & \textcolor{green}{\Checkmark} & \textcolor{green}{\Checkmark} & \textcolor{green}{\Checkmark} & \textcolor{green}{\Checkmark} \\
        \bottomrule
  \end{tabular}
\end{table*}

\textbf{Graph Storage}.

\textbf{Graph Matching Algorithms}.
Generally speaking, there are two kinds of graph isomorphism algorithms:

The first is the tree-based searching method.
Inspired by Ullmann's backtracking algorithm~\cite{DBLP:journals/jacm/Ullmann76},
many enhancements have been proposed providing different searching order, filter rules, and indexes~\cite{DBLP:journals/pami/CordellaFSV04,DBLP:journals/pvldb/ShangZLY08,DBLP:conf/sigmod/HeS08,DBLP:conf/sigmod/HanLL13,DBLP:journals/pvldb/LeeHKL12}.
However, these algorithms are not suitable for out-of-core systems as random disk seeks are unavoidable.

The second is the join-based algorithm~\cite{DBLP:journals/pvldb/LaiQLC15,DBLP:journals/pvldb/QiaoZC17,DBLP:journals/pvldb/SunWWSL12,DBLP:journals/pvldb/MhedhbiS19,DBLP:journals/pvldb/MhedhbiS19}.
These algorithms split the pattern graph into smaller units,
and materialize the intermediate results.
The final result is obtained by joining on these intermediate results.
SeqStar also belongs to this category.

The most fundamental problem of a join-based algorithm is to determine the basic matching unit.
Graphflow~\cite{DBLP:journals/pvldb/MhedhbiS19} uses a cost-based optimizer to perform worst case optimal join or binary join on edges.
However, edges contain very little topological information and may generate unnecessary matchings at the first stage.
STwig~\cite{DBLP:journals/pvldb/SunWWSL12} and TwinTwig~\cite{DBLP:journals/pvldb/LaiQLC15} use star-like structures.
However, their star decomposition algorithm may discard useful filtering information.
Whereas SeqStar always keep all the helpful information and we can apply predicate pushdown to boost the star matching process.

VCBC~\cite{DBLP:journals/pvldb/QiaoZC17} is the first intermediate result compression algorithm for graph isomorphism problem.
Based on VCBC, they provide a graph isomorphism algorithm by matching crystals (clique-like structure).
And the compression ratio of VCBC drops as the crystals become complex.
In contrast, SeqStar stores only the intermediate of star matching, and keeps the compression ratio at a high level.
%% \subsection*{In-memory Methods}
%% Most of the early work assumes that the data graph and indices are fit in the main memory of a single machine.
%% Sparked by Ullmann's backtracking algorithm~\cite{DBLP:journals/jacm/Ullmann76},
%% many subgraph matching algorithms have been proposed using different searching order, filter rules, and neighborhood indices~\cite{DBLP:journals/pami/CordellaFSV04,DBLP:journals/pvldb/ShangZLY08,DBLP:conf/sigmod/HeS08,DBLP:conf/sigmod/HanLL13,DBLP:journals/pvldb/LeeHKL12}.
%% These algorithms usually use a DFS-style tree-based graph exploration to search the matchings without materializing intermediate results.
%% However, these single machine in-memory algorithms are no longer suitable for nowadays billion-node graphs.

%% To address the scalability problem of single machine in-memory algorithms,
%% many distributed subgraph matching algorithms have been proposed~\cite{DBLP:journals/pvldb/SunWWSL12,DBLP:conf/sigmod/ShaoCCMYX14,DBLP:journals/pvldb/LaiQLC15,DBLP:journals/pvldb/LaiQLZC16,DBLP:conf/cloud/SerafiniMS17}.
%% Because the vertices of the data graph are scattered among machines,
%% these algorithms usually match smaller patterns and get the final result by join operation.
%% For example, Sun et al.~\cite{DBLP:journals/pvldb/SunWWSL12} introduce a star-like basic matching unit called STwig,
%% and implement their subgraph matching algorithm on top of the Trinity~\cite{shao2012the} memory cloud.
%% Lai et al.~\cite{DBLP:journals/pvldb/LaiQLC15} propose TwinTwig join using MapReduce,
%% where a TwinTwig is either a single edge or two incident edges of a vertex.
%% The SEED~\cite{DBLP:journals/pvldb/LaiQLZC16} algorithm use both star and clique as the join units,
%% and use clique compression technique to further improve the performance.
%% However, these distributed algorithms still suffer from severe memory crisis,
%% because the size of partial results grow exponentially with respect to the size of the date graph.
%% Moreover, they must be transferred to other machines before join,
%% which is the most expensive operation in a parallel system such as MapReduce.

%% Besides, the optimization of a subgraph matching algorithm relies heavily on the underlying graph model:

%% Unlabeled undirected simple graph is perhaps the simplest graph model,
%% which can be viewed as a special case of property graph with all the vertices and edges have the same label and have no multi-edges.
%% Some authors distinguish this kind of graphs from others and designate the matching problem of this kind of graph as \emph{subgraph listing}~\cite{DBLP:conf/sigmod/ShaoCCMYX14,DBLP:journals/jacm/Ullmann76,DBLP:conf/sigmod/ShaoCCMYX14,DBLP:journals/pvldb/LaiQLC15,DBLP:conf/sigmod/KimLBHLKJ16,DBLP:journals/pvldb/LaiQLZC16,DBLP:journals/pvldb/QiaoZC17}.
%% CBF~\cite{DBLP:journals/pvldb/QiaoZC17} is the state-of-the-art subgraph listing algorithm,
%% which decompose the pattern graph into a several basic structures called \emph{crystals},
%% and match these basic units with partial results compressed by the VCBC algorithm.
%% However, it is unable to support general property graph because CBF relies on clique listing to match crystals,
%% which implies the equivalence of vertices in a clique (complete graph) and is not the case of property graph model because of labels and direction of edges.

%% Another widely studied graph model is vertex-labeled undirected simple graph~\cite{DBLP:journals/pvldb/ShangZLY08,DBLP:journals/pvldb/SunWWSL12,DBLP:conf/sigmod/HanLL13,DBLP:conf/cloud/SerafiniMS17,DBLP:conf/sigmod/DiasTGM019}.
%% Turbo\textsubscript{ISO}~\cite{DBLP:conf/sigmod/HanLL13}, for example, is turbocharged by the concept of \emph{neighborhood equivalence class} (NEC).
%% It outperforms other competitors by safely avoid the permutation of all possible vertices in the same NEC\@.
%% A NEC is a set of vertices in the pattern graph, where every vertex has the same label and the same set of neighbors.
%% However, things become more complex and make it not suitable for the property graph model.
%% Because one has to check the labels of vertices, labels of edges, directions of edges in order to test the isomorphism of a property graph, and the real-world multigraphs make life even harder.
%% \subsection*{Out-of-core Methods}
%% Many out-of-core triangle enumeration algorithms have been proposed~\cite{DBLP:conf/kdd/ChuC11,DBLP:conf/osdi/KyrolaBG12,DBLP:conf/sigmod/HuTC13,DBLP:conf/sigmod/KimHLPY14}.
%% However, all these algorithms only deal with triangulation, a special case of the graph matching problem.
%% Recently, \textsc{DualSim}~\cite{DBLP:conf/sigmod/KimLBHLKJ16} take a further step and is able to match general unlabeled undirected graphs.
%% To avoid the materialization of intermediate results,
%% it fixes the data vertices by fixing a set of disk pages and then find all matchings in these pages.
%% Apparently, every page of the data graph must be swapped in/out many times in order to get the final result,
%% which lead to severe I/O cost.
%% In contrast, our approach will load the pages sequentially at most once,
%% and we can also use the compressed partial results to boost afterward queries.
