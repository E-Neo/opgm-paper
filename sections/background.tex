\section{Background}\label{sec:background}
A \emph{property graph} is a directed graph with labelled vertices and edges. It may also have self-loops i.e. edges connecting back to their source vertices.

\begin{definition}[Property Graph~\cite{DBLP:journals/csur/AnglesABHRV17}]
  A property graph $G$ is a tuple $(V, E, \rho, \lambda, \sigma)$, where:
  \begin{enumerate}[noitemsep,label={(\arabic*)}]
  \item $V$ is a finite set of vertices.
  \item $E$ is a finite set of edges.
  \item $\rho: E \rightarrow (V \times V)$ is a total function.
    Intuitively, $\rho(e) = (v_1, v_2)$ indicates that $e$ is a directed edge from $v_1$ to $v_2$.
  \item $\lambda :(V \cup E) \rightarrow L$ is a total function where $L$ is a set of labels.
    Intuitively, if $v \in V, \rho(v) = l$ (respectively, $e \in E, \rho(e) = l$),
    $l$ is the label of vertex $v$ (respectively, edge $e$).
  \end{enumerate}
\end{definition}
A property graph $G$ is denoted as $(V(G), E(G), \rho_G, \lambda_G)$.
Note that the total function $\rho_G$ is necessary because of the self-loops.
\subsection{Property Graph Matching Problem}
\begin{definition}[Subgraph]
  A property graph $H$ is called a subgraph of a property graph $G$, denoted as $H \subseteq G$, if
  $V(H) \subseteq V(G)$, $E(H) \subseteq E(G)$, $\rho_H$ is a restriction of $\rho_G$, and $\lambda_H$ is a restriction of $\lambda_G$.
\end{definition}
Note that the restriction means the sub-graph $H$ must contain all edges in $G$ connecting all vertices in $H$.

%Intuitively, a subgraph is a ``smaller'' property graph in the original graph.
%Let $G$ be any property graph, and let $V_H \subseteq V(G)$, then the \emph{induced subgraph} $G[V_H]$ is the graph $H$ whose vertex set is $V_H$ and whose edge set consists of all of the edges in $E(G)$ that have both endpoints in $V_H$.
\begin{definition}[Property Graph Isomorphism]
  Two property graphs $G$ and $H$ are isomorphic, denoted as $G \cong H$,
  if there exists bijections $\theta: V(G) \rightarrow V(H)$ and $\phi: E(G) \rightarrow E(H)$ such that
  $\rho_G(e) = (u, v)$ if and only if $\rho_H(\phi(e)) = (\theta(u), \theta(v))$,
  $\lambda_G(v) = \lambda_H(\theta(v))$ for all $v \in V(G)$
  and $\lambda_G(e) = \lambda_H(\phi(e))$ for all $e \in E(G)$;
  Such a pair of mappings is called an isomorphism between $G$ and $H$.
\end{definition}
Given a large graph and a (much smaller) pattern graph,
the property graph isomorphism problem is to find all subgraphs of the large graph that are isomorphic to the pattern graph.
Unlike the simple graph model, the property graph isomorphism problem is much more complicated.
As the vertices and edges are labeled in a property graph, we need to check not only the topological structure,
but also the labels to determine whether two property graphs are isomorphic.
And the existence of multi-edge connecting two vertices makes it even harder to check the isomorphism.
\begin{definition}[Property Graph Matching]\label{def:property_graph_matching}
  Given a data property graph $D$, a pattern property graph $P$ and a searching condition $\psi: \mathcal{G} \rightarrow \mathcal{B}$ with $\mathcal{G}$ the set of property graphs and $\mathcal{B}$ the set of Boolean values,
  the property graph matching problem is to report the set $\mathcal{I} = \{H | H \subseteq D, H \cong P, \psi(H) = true\}$.
\end{definition}
The property graph matching problem extends the property graph isomorphism problem by introducing the searching conditions (WHERE clause).
Authors of previous works usually omit the searching condition $\psi$ in their definition of graph matching~\cite{DBLP:conf/sigmod/ShaoCCMYX14,DBLP:journals/pvldb/LaiQLC15,DBLP:conf/sigmod/KimLBHLKJ16,DBLP:journals/pvldb/QiaoZC17}.
And they adopt a loosely related technique called \emph{symmetry-breaking}~\cite{DBLP:conf/recomb/GrochowK07},
which can be viewed as a special case of WHERE clause by providing a partial order on $V(P)$ after exploiting the automorphism of $P$.
However, as the WHERE clause is a necessary part of a graph query language,
users of a real-world graph database usually provide their self-defined searching condition $\psi$ to filter out unnecessary matchings not only symmetry-breaking conditions.
And we also found that it can be decomposed and pushed down to lower phase to boost the evaluation of graph matching (Section~\ref{sec:match_optimize}).
