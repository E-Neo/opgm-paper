\section{Background}\label{sec:background}
This section introduces the formal definition of property graphs, and then discusses the property graph matching problem.
\subsection{Property Graph Model}
A \emph{property graph} is a directed vertex-labeled edge-labeled multigraph with self-edges,
and key-value properties are stored on vertices and edges.
We now provide the formal definition of a property graph.
\begin{definition}[Property Graph~\cite{DBLP:journals/csur/AnglesABHRV17}]
  A property graph $G$ is a tuple $(V, E, \rho, \lambda, \sigma)$, where:
  \begin{enumerate}[noitemsep,label={(\arabic*)}]
  \item $V$ is a finite set of vertices.
  \item $E$ is a finite set of edges such that $V$ and $E$ have no elements in common.
  \item $\rho: E \rightarrow (V \times V)$ is a total function.
    Intuitively, $\rho(e) = (v_1, v_2)$ indicates that $e$ is a directed edge from $v_1$ to $v_2$.
  \item $\lambda :(V \cup E) \rightarrow L$ is a total function where $L$ is a set of labels.
    Intuitively, if $v \in V, \rho(v) = l$ (respectively, $e \in E, \rho(e) = l$),
    then $l$ is the label of vertex $v$ (respectively, edge $e$).
  \item $\sigma: (V \cup E) \times Prop \rightarrow Val$ is a partial function with $Prop$ a finite set of properties and $Val$ a set of values.
    Intuitively, if $v \in V, p \in Prop, \sigma(v, p) = s$ (respectively, $e \in E, p \in Prop, \sigma(e, p) = s$),
    then $s$ is the value of property $p$ for vertex $v$ (respectively, edge $e$) in the property graph $G$.
  \end{enumerate}
\end{definition}
For simplicity, in this paper, we do not discuss the properties i.e., $\sigma$ in $G$,
because similar techniques can be used as processing the labels $\lambda$.
Thus, the property graph $G$ can be denoted by $(V(G), E(G), \rho_G, \lambda_G)$.
Please note that the total function $\rho_G$ is necessary, in general,
we cannot identify an edge simply by the starting and ending vertices such as $(u_1, u_2)$ as can be done in the simple graph model,
because multiple edges may appear between the two vertices.
However, we may use the $(u_1, u_2)$ notation if all we care about is that there exist at least one edge between $u_1$ and $u_2$.
\begin{definition}[Vertex Cover]
  A vertex cover $V_c$ of a property graph $G$ is a subset of $V(G)$ such that
  $\forall e \in E(G), \rho_G(e) = (u, v) \implies u \in V_C \lor v \in V_C$.
\end{definition}
\subsection{Property Graph Matching Problem}
\begin{definition}[Subgraph]
  A property graph $F$ is called a subgraph of a property graph $G$, written $F \subseteq G$, if
  $V(F) \subseteq V(G)$, $E(F) \subseteq E(G)$, $\rho_F$ is a restriction of $\rho_G$, and $\lambda_F$ is a restriction of $\lambda_G$.
\end{definition}
Let $G$ be any property graph, and let $S \subseteq V(G)$, then the \emph{induced subgraph} $G[S]$ is the graph whose vertex set is $S$ and whose edge set consists of all of the edges in $E(G)$ that have both endpoints in $S$.
\begin{definition}[Property Graph Isomorphism]
  Two property graphs $G$ and $H$ are isomorphic, written $G \cong H$,
  if there exists bijections $\theta: V(G) \rightarrow V(H)$ and $\phi: E(G) \rightarrow E(H)$ such that
  $\rho_G(e) = (u, v)$ if and only if $\rho_H(\phi(e)) = (\theta(u), \theta(v))$,
  $\lambda_G(v) = \lambda_H(\theta(v))$ for all $v \in V(G)$
  and $\lambda_G(e) = \lambda_H(\phi(e))$ for all $e \in E(G)$;
  Such a pair of mappings is called an isomorphism between $G$ and $H$.
\end{definition}
The bijection $\theta: V(G) \rightarrow V(H)$ is the key in the definition of property graph isomorphism,
because the bijection $\phi: E(G) \rightarrow E(H)$ is straightforward if $\theta$ is fixed.
However, due to automorphism, where an \emph{automorphism} of a graph is an isomorphism of the graph to itself,
the bijection $\theta$ may not be unique.
\begin{definition}[Property Graph Matching]\label{def:property_graph_matching}
  Given a data property graph $D$, a pattern property graph $P$ and a searching condition $\psi: PG \rightarrow B$ with $PG$ the set of property graph and $B$ the set of Boolean values,
  the property graph matching problem is to report the set $\mathcal{I} = \{F | F \subseteq D, F \cong P, \psi(F) = true\}$.
\end{definition}
Authors of previous works usually omit the searching condition $\psi$ in their definition of graph matching~\cite{DBLP:conf/sigmod/ShaoCCMYX14,DBLP:journals/pvldb/LaiQLC15,DBLP:conf/sigmod/KimLBHLKJ16,DBLP:journals/pvldb/QiaoZC17}.
And they adopt a loosely related technique called \emph{symmetry-breaking}~\cite{DBLP:conf/recomb/GrochowK07},
which ensures there is a unique bijection $\theta: V(P) \rightarrow V(F)$ by providing a partial order on $V(P)$ after exploiting the automorphism of $P$.
However, as we have stated before, the WHERE clause is a ubiquitous part of the query language of a graph database.
Users of a real-world graph database usually provide their self-defined searching condition $\psi$ to filter out unnecessary matchings not only symmetry-breaking conditions.
Thus, the searching condition we defined here can be viewed as a super set of symmetry-breaking.
We add the searching condition in the definition because it is actually a part of the property graph matching problem,
and we also found that it can be decomposed and pushed down to lower phase to boost the evaluation of graph matching (Section~\ref{sec:framework}).

A property graph is always directed.
However, in some cases such as friendship, there is no need to pay attention to the directions of the edges.
In order to support this kind of relationship, a naive approach is to add a duplicate edge in opposite direction for each edge in the data graph.
More elegantly, we allow the pattern graph $P$ to contain undirected edges.
Users can simply ignore the direction by providing undirected edges in $P$ like in industrial graph databases such as Neo4j.
