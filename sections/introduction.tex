\section{Introduction}
Graph matching
%~\footnote{There are two kinds of graphs in a graph matching problem: one is the large \emph{data graph}, the other is the much smaller \emph{pattern graph}} 
is one of the most important applications of graph databases. In graph matching, a relative small \emph{pattern graph} is used to match subgraphes of a relative large \emph{data graph}.
It is widely used in many fields,
such as Twitter's recommendation systems~\cite{DBLP:journals/pvldb/GuptaSGGZLL14,DBLP:journals/pvldb/SharmaJBLL16},
electronic computer-aided design~\cite{DBLP:conf/dac/OhlrichEGS93},
and protein-protein interaction (PPI) networks~\cite{milenkovic2008uncovering}.
Nowadays, users of industrial graph databases such as Neo4j\footnote{\url{https://neo4j.com}}
can easily model data as property graphs and expressing their queries via the Cypher~\cite{DBLP:conf/sigmod/FrancisGGLLMPRS18} query language.
Although it's convenient, the matching process of these industrial graph databases are time and resource consuming.
Many novel subgraph matching algorithms have been proposed~\cite{DBLP:journals/pvldb/SunWWSL12,DBLP:conf/sigmod/HanLL13,DBLP:conf/sigmod/ShaoCCMYX14,DBLP:conf/cloud/SerafiniMS17,DBLP:journals/pvldb/QiaoZC17,DBLP:conf/sigmod/DiasTGM019}, with even orders of magnitude of speedup compared to the industrial graph databases.
However, there are two gaps that hinder these algorithms from being widely adopted in the real-world scenarios:
(1) the gap between the complexity of real-world queries and the simplicity of graphs that the existing algorithms can handle;
(2) the huge memory and high-speed network requirements for the existing algorithms to query on large graphs and the limitation of resource budget.

The property graph is the de facto graph model for real-world graph matching problems.
A property graph is a directed graph with labels attached to vertices and edges.
There may be multiple edges connecting two vertices and edges for self-loop.
Moreover, users usually provide extra searching conditions via the WHERE clause~\cite{DBLP:journals/csur/AnglesABHRV17}. However,  current studies mostly deal with simple graph models such as ignoring the directions, labels, or multi-edges. WHERE clause is not considered either~\cite{DBLP:journals/pvldb/SunWWSL12,DBLP:conf/sigmod/HanLL13,DBLP:conf/sigmod/KimLBHLKJ16,DBLP:journals/pvldb/QiaoZC17,DBLP:journals/pvldb/MhedhbiS19}. The optimizations aimed at simple graphs usually heavily rely on the equivalence of vertices~\cite{DBLP:conf/sigmod/HanLL13,DBLP:journals/pvldb/QiaoZC17}, which is harder to find in a property graph because of the complexity of directions and labels. Moreover, in the implementation level, the storage engine designed for simple graph models encounters excessive random disk seeks when matching a real-world property graph.

%This limitation is not only a matter of engineering,
%but also a serious algorithm problem:
%(1) the storage engine designed for simple graph may encounter superfluous random disk reads when matching a property graph;
%(2) the optimizations aimed at simple graph usually rely heavily on the equivalence of vertices~\cite{DBLP:conf/sigmod/HanLL13,DBLP:journals/pvldb/QiaoZC17}, which is harder to find in a property graph because of the complexity of directions and labels.

Existing algorithms rely on main memory for their computation. It is extremely uneconomic for property graph matching problem. The memory needs to hold not only the graph data but also the intermediate results which grow exponentially with respect to the size of graph data. Our experiment shows that a graph with $6.9 \times 10^{7}$ edges could generate $1.7 \times 10^{13}$ rows of matching results.
%(1) The memory needs to hold not only the data graph, but also the intermediate result which grows exponentially with respect to the size of the data graph, as our experiment shows that a graph with $6.9 \times 10^{7}$ edges could generate $1.7 \times 10^{13}$ rows of matching results.
Existing systems~\cite{DBLP:conf/sosp/TeixeiraFSSZA15,DBLP:conf/sigmod/DiasTGM019,DBLP:journals/pvldb/MhedhbiS19} require hundreds of GB of memory to process such graphs.
%(2) For distributed approaches, on the one hand it is hard to partition the graph across cluster machines to minimize the communication cost~\cite{DBLP:journals/im/LeskovecLDM09} and have performance problems~\cite{DBLP:conf/sigmod/KimLBHLKJ16},
%one the other hand it is inconvenient and costly for the end user to maintain the cluster nodes~\cite{DBLP:conf/osdi/KyrolaBG12}.

Therefore, we propose SeqStar, a high performance out-of-core property graph matching system for real-world graphs, which scans the disk sequentially by matching stars (a star contains a root vertex and some leaves which are the neighbors of the root).

%Therefore, in order to match up with the need of real-world problems,a high performance disk-based property graph matching system is desirable.

\subsection*{Contributions}
There are two fundamental components in SeqStar:
the graph storage engine and the graph matching engine. Both are designed specificly to deal with the graph machine problem efficiently.

The storage engine in SeqStar uses the \emph{vertex-centric storage model} which stores the information related to a specific vertex together.
Such a model can eliminate the random access problem in compressed sparse row (CSC) and compressed sparse column (CSR) which store in/out-edges separately~\cite{DBLP:conf/sc/PearceGA10,DBLP:conf/osdi/KyrolaBG12}. %这里加上csc和csr的全称,并加上引用

For the graph matching engine, SeqStar uses a star decomposition based algorithm to run graph matching. %这里加一句如何把原图分解为星型子图
We save all useful filtering information in each star subgraph after decomposition. To reduce the amount of memory that would be used as the storage of intermediate results, a novel compression method, which postponing Cartesian production and digging equivalence classes, is used for star matching results. SeqStar's efficient and scalable pipeline join algorithm is able to process the compressed data. To the best of our knowledge, SeqStar is the first system that integrates the optimization of WHERE clauses in the graph matching algorithm, by pushing the predicates down to the star matching process.

%To reduce memory usage, we make the following contributions:
%(1) a novel star decomposition algorithm that keeps all the useful filtering information in stars,
%(2) a novel compression algorithm for stars' matching results by postponing Cartesian production and digging equivalence classes,
%(3) an efficient and scalable pipeline join algorithm that is able to process the compressed data directly.

%Moreover, to the best of our knowledge, we are the first one to integrate the optimization of WHERE clauses in the graph matching algorithm, by pushing the predicates down to the star matching process.
%% The conventional graph storage method is the compressed sparse column (CSC) and the compressed sparse row (CSR) format~\cite{DBLP:conf/osdi/KyrolaBG12},
%% which stores the in/out-edges separately for each vertex.
%% However, because of the multi-edges in property graphs, the in/out-edges have to be searched many times to check the matching of a vertex, and result in excessive random disk seeks.
%% To address this challenge, we propose the \emph{vertex-centric storage model} that stores all the necessary information together with the vertices, such that the vertices could be matched in a single sequential scan.

%% Generally speaking, there are two kinds of graph isomorphism algorithm,
%% differing on whether intermediate results are materialized.
%% The first is the backtracking tree-searching method~\cite{DBLP:journals/jacm/Ullmann76,DBLP:journals/pvldb/LeeHKL12,DBLP:conf/sigmod/HanLL13,DBLP:conf/sigmod/KimLBHLKJ16},
%% which does not generate intermediate results but have scaling problems~\cite{DBLP:conf/cloud/SerafiniMS17}.
%% The second is the join-based algorithm~\cite{DBLP:journals/pvldb/LaiQLC15,DBLP:journals/pvldb/QiaoZC17,DBLP:journals/pvldb/SunWWSL12,DBLP:journals/pvldb/MhedhbiS19},
%% which decomposes the pattern graph into smaller matching unit and materialize the intermediate results,
%% and the final result is obtained by joining on these intermediate results.
%% Because of the notorious poor locality of graphs,
%% enormous amount of random disk accesses would be encountered for an out-of-core tree-searching approach.
%% Based on this observation, this paper designs a join-based property graph matching algorithm.

%% The most fundamental problem of a graph matching algorithm is to determine the basic matching unit.
%% Edges are the simplest matching units, however, intermediate results much larger than the data graph would be generated and result in costly join operations.
%% To avoid excessive joins, authors use more complex structures i.e.,
%% frequent subgraphs, multi-hop edges, or cliques as their matching unit~\cite{DBLP:conf/sigmod/HeS08,DBLP:conf/edbt/ZhangLY09,DBLP:journals/pvldb/QiaoZC17}, however,
%% these algorithms rely on super-linear indices~\cite{DBLP:journals/pvldb/SunWWSL12}.
%% To address this challenge, we make a balance by choosing stars as our basic unit.
%% And our star decomposition algorithm is enhanced such that the stars keep as much filtering information as possible.
%% Our experiment shows that this enhancement could reduce the intermediate results by up to $43\%$ of the existing works.

%% However, the stars' matching results could still be very large because the matching results grow exponentially with respect to the data graph,
%% e.g., a 129M graph could easily eat up hundreds or even thousands GB of memory to store the intermediate results.
%% Inspired by VCBC~\cite{DBLP:journals/pvldb/QiaoZC17}, we develop a novel compression algorithm for star' matching result by postponing the costly Cartesian product and digging the equivalence classes in the leaves of the star.
%% The compression ratio is quite impressive, as high as $10^8$,
%% and we find that the size of the compressed star's matching results takes less than $23\%$ the space to store the graph,
%% which can even fit into the main memory of a laptop.
