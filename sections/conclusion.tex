\section{Conclusion}\label{sec:conclusion}
\textcolor{red}{This paper proposes SeqStar, an out-of-core property graph matching system.}
%This paper proposes SeqStar, which is the first out-of-core property graph matching system.
SeqStar uses the vertex-centric storage engine to store the information related to one vertex together.
This storage engine avoids the random disk access problemand provides the convenient iterators for programs to visit the required vertices.
Small indexes are designed to further minimize the number of IO operations.
For the graph matching engine,
SeqStar uses a novel star decomposition algorithm that preserves as much filtering information as possible in decomposed stars from the pattern graph.
Predicate pushdown is applied to further reduce unnecessary matchings.
In order to reduce the memory usage, SeqStar compresses the intermediate results by postponing Cartesian product and combining equivalent vertices.
And SeqStar performs parallel pipeline join on the compressed data to avoid unnecessary intermediate results.
Experimental results demonstrate that SeqStar outperforms existing work and is able to run more complex property graph queries with less memory consumption.


Currently, SeqStar lacks the support for dynamic graphs.
This limitation gives us directions for future work,
and we are now working on implementing the dynamic vertex-centric storage model to support dynamic graphs.
% 这里扯了一下 future work,看其他的 VLDB 都要在这说,甚至有的这整章就是 future work
