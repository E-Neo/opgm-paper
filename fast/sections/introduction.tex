\section{Introduction}
Graph matching is one of the most important applications of graph databases.
Given a pattern graph $P$ and a data graph $D$, the graph matching problem is to find all the subgraphs of $D$ that are isomorphic to $P$.
It is widely used in many fields,
such as Twitter's recommendation systems~\cite{DBLP:journals/pvldb/GuptaSGGZLL14,DBLP:journals/pvldb/SharmaJBLL16},
electronic computer-aided design~\cite{DBLP:conf/dac/OhlrichEGS93},
protein-protein interaction (PPI) networks~\cite{milenkovic2008uncovering},
and epidemic emergency management~\cite{info:doi/10.2196/26836}.

The workloads of graph matching present great challenges for system designers.
(1) The property graphs are directed and labeled, and multiple edges may present between a pair of vertices.
In order to run the graph matching algorithm with a high performance, the underlying graph storage must support these features efficiently.
(2) Many graphs exhibit a power-law distribution on the degree of vertices.
Cutting a graph into balanced communities to optimize locality and network communications is difficult~\cite{DBLP:journals/im/LeskovecLDM09}.
(3) The matching results grow exponentially with respect to the size of the data graph.
Qiao et al\@. found that a 390-megabyte data graph can result in 25-terabyte intermediate results, even though they are compressed~\cite{DBLP:journals/pvldb/QiaoZC17}.
As graphs are much larger nowadays, the system must handle the intermediate results properly.
As a result, it is hard to address the problem efficiently via distributed memory systems.

Existing graph matching algorithms can be categorized into \emph{exploration-based} and \emph{join-based} methods based on their graph scanning patterns~\cite{DBLP:journals/pvldb/SunSC0H20}.
The exploration-based algorithms~\cite{DBLP:journals/jacm/Ullmann76,DBLP:journals/pami/CordellaFSV04,DBLP:journals/pvldb/ShangZLY08,DBLP:conf/sigmod/HeS08,DBLP:conf/sigmod/HanLL13,DBLP:journals/pvldb/ZhaoH10,DBLP:journals/pvldb/LeeHKL12}
iteratively extend the intermediate results by exploring and checking vertices in $D$ according the visiting order of vertices in $P$.
In contrast, the join-based algorithms~\cite{DBLP:journals/pvldb/LaiQLC15,DBLP:journals/pvldb/QiaoZC17,DBLP:journals/pvldb/SunWWSL12,DBLP:journals/pvldb/MhedhbiS19,DBLP:journals/pvldb/LinMPS16,DBLP:journals/pvldb/AmmarMSJ18} split the pattern graph $P$ into smaller matching units, scan the data graph $D$ and match the units, then obtain the final results by joining on the intermediate data.
Clearly, both exploration-based and join-based algorithms rely on the fast neighborhood checking ability provided by the underlying graph storage.

Most graph storage methods are based on the technologies to store sparse matrix.
The basic idea is to store the edges, and provide fast in/out-edges accesses by partitioning on the destination/source vertices.
For example, compressed sparse row (CSR)~\cite{1447941} stores the edges consecutively, where edges are partitioned by sources.
Given a vertex $v$, CSR returns a (position, offset) pair to locate the out-edges of $v$.
Compressed sparse column (CSC) can be viewed as the transpose of CSR, and it provides fast access to in-edges of a given vertex.
Neo4j~\cite{Neo4j} has a linked list based implementation to store the edges, and can visit both in/out-edges efficiently by storing the (reference of) edges twice.

The traditional in/out-edges access pattern works fine if only vertices were not allowed to follow each other.
However, the 2-cycle ($u_1 \leftrightarrows u_2$) structures are prevalent among social networks and web graphs (Table TODO).
For these graphs, the matching process will be very inefficient if the graph storage adopts the in/out-edge splitting model:
(1) For each vertex $v$ in the data graph $D$, one have to scan the out-edges of $v$, and explore every destination vertex to check whether it has an incoming edge;
Or (2) use a join-based algorithm to match $u_1 \rightarrow u_2$ and $u_1 \leftarrow u_2$ separately, then join them together.
Both of them run in $O(m^2)$ time, where $m$ is the number of edges in $D$.
Note that simply save the edges twice as in/out-edges will not improve the performance.
And it is even worse if more 2-cycle structures present in the pattern graph.

In this paper, we present SeqStar, an efficient on-disk index for graph matching problems.
Prior indexes can be categorized into two aspects:
(1) \emph{adjacency indexes}~\cite{bonifati2018querying} use different partitioning techniques to provide constant time access to neighbors vertices.
For example, CSR partitions the edges by source vertices, and some systems provide secondary level partitioning in these structure by edge labels~\cite{DBLP:conf/icde/MhedhbiGKS21}.
(2) \emph{structural indexes}~\cite{DBLP:conf/europar/SumrallFPSVVW16,DBLP:journals/pvldb/LaiQLZC16,DBLP:journals/pvldb/ZhuQLYHY11} index more complex structures such as paths, cliques, or frequent subgraphs.
However, these indexes are not designed for the complex pattern matching problems as we stated before.
Moreover, many prior structural indexes require super-linear indexing space~\cite{DBLP:journals/pvldb/SunWWSL12}.

To close the gap between complex matching problems and the unsatisfactory graph storage,
We propose a new \emph{vertex-centric} storage model for SeqStar and introduces a novel access pattern on top of it.
Instead of storing the in/out-edges \emph{separately}, SeqStar stores the relationships \emph{together} with the neighbor vertices.
SeqStar advocates a new paradigm that neighbors are accessed via vertices, rather than in/out-edges.
We provide two kinds of iterators to visit the data graph:
(1) \textsc{VertexIter} visits the data vertices, and (2) \textsc{NeighborIter} visits the neighbor vertices of a given data vertex.
In/out-edge topologies are associated with the neighbor vertex.
Therefore, 2-cycles or more complex multiple edges can be checked within a single sequential scan ($O(m)$).
Experiments show that TODO\@.

For large power-law graphs such as social network, the degree of celebrities or trending topics can be tremendous.
Reducing the amount of I/O for the neighborhood checking process of the celebrities directly improves performance.
SeqStar introduces \emph{local indexes} to partition the neighbor vertices.
Instead of storing the indexes separately by references, the local indexes are stored together with the corresponding vertices.
Therefore, the on-disk layout of the local indexes has a better locality.
Experiments show that TODO\@.

On top of SeqStar's vertex-centric graph index, we take more steps and build a star-based graph matching engine.
(1) Prior star-based graph matching algorithms such as STwig~\cite{DBLP:journals/pvldb/SunWWSL12} and TwinTwig~\cite{DBLP:journals/pvldb/LaiQLC15} also decompose the pattern graph into stars.
However, we found that these systems will lose valuable structural information in their stars, and result in many unnecessary intermediate results.
To address this problem, we introduce a new star decomposition algorithm that will preserve all the necessary filtering information.
Experiments show that TODO\@.
(2) We adopt and improve the vertex cover based compression algorithm (VCBC)~\cite{DBLP:journals/pvldb/QiaoZC17} to store the star matching results as \emph{SuperRows}.
We found that VCBC still suffer from super-linear intermediate data explosion, where MB-sized graph may generate TB-sized intermediate data.
In contrast, SuperRow ensures the intermediate data always grows \emph{linearly}.
SeqStar obtains the final results by applying parallel pipeline join on the SuperRows.
Experiments show that TODO\@.
